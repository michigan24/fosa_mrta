% coding:utf-8

%----------------------------------------
%FOSAPHY, a LaTeX-Code for a summary of modern control theory
%Copyright (C) 2015, Mario Felder & Michi Fallegger

%This program is free software; you can redistribute it and/or
%modify it under the terms of the GNU General Public License
%as published by the Free Software Foundation; either version 2
%of the License, or (at your option) any later version.

%This program is distributed in the hope that it will be useful,
%but WITHOUT ANY WARRANTY; without even the implied warranty of
%MERCHANTABILITY or FITNESS FOR A PARTICULAR PURPOSE.  See the
%GNU General Public License for more details.
%----------------------------------------

\section{Transformation}
\subsection{Transformation in Regelungsnormalform (Steuernormalform)}
Ein Übertragungssystem 
\[ \dot{x}=A\cdot x +  b \cdot u \]
wird mit der Transformation
\[ z=T_R\cdot x \]
in die Regelungsnormalform
\[ \dot{x}=A_R\cdot x +  b_R \cdot u \]
überführt. Die Matrizen werden folgendermassen transformiert:
\[\begin{aligned}
	A_R &= T_R\cdot A \cdot {T_R}^{-1}	\\
	b_R &= T_R\cdot b	\\	
	{c^T}_R &= c^T\cdot {T_R}^{-1}	\\	
	d_R &= d
\end{aligned}\]
Die Transformationsmatrix $T_R$ ($n\times n$) wird wie folgt berechnet:
\[ \renewcommand\arraystretch{1.3}
	T_R= \begin{bmatrix}
		q_{S_n}^T \\
		q_{S_n}^T \cdot A \\
		\vdots \\
		q_{S_n}^T \cdot A^{n-1}	
	\end{bmatrix}
\]
wobei $q_{S_n}^T$ der $n$-te Zeilenvektor der inversen Steuerbarkeitsmatrix ${Q_S}^{-1}$ ist
\[
	{Q_S}^{-1} = \begin{bmatrix}
		q_{S_{11}} & q_{S_{12}} & \ldots & q_{S_{1n}}\\
		q_{S_{21}} & q_{S_{22}} & \ldots & q_{S_{2n}}\\
		\vdots		& \vdots	 & \ddots & \vdots\\
		q_{S_{n1}} & q_{S_{n2}} & \ldots & q_{S_{nn}}\\	 
	\end{bmatrix} \Rightarrow
	q_{S_n}^T=
	\begin{bmatrix}
		q_{S_{n1}} & q_{S_{n2}} & \ldots &	q_{S_{nn}}
	\end{bmatrix}	
\]

\subsection{Transformation auf Beobachtungsnormalform}
Ein Übertragungssystem 
\[ \dot{x}=A\cdot x +  b \cdot u \]
wird mit der Transformation  
\[ z=T_B\cdot x \]
in die Beobachtungsnormalform
\[ \dot{x}=A_B\cdot x +  b_B \cdot u \]
überführt. Die Matrizen werden folgendermassen transformiert:
\[\begin{aligned}
	A_B &= T_B^{-1}\cdot A \cdot {T_B}	\\	
	b_B &= T_B\cdot b	\\	
	{c^T}_B &= c^T\cdot {T_B}^{-1}	\\	
	d_B &= d
\end{aligned}\]
Die Transformationsmatrix $T_B$ ($n\times n$) wird wie folgt berechnet:
\[
	T_B=
	\begin{bmatrix}
		q_{B_n} & A\cdot q_{B_n} &  \ldots & A^{n-1}\cdot q_{B_n} & \\
	\end{bmatrix}	\\
\]
wobei $q_{B_n}$ der $n$-te Spaltenvektor der inversen Beobachtungsmatrix ${Q_B}^{-1}$ ist.
\[
	{Q_B}^{-1} = \begin{bmatrix}
	 q_{B_{11}} & q_{B_{12}} & \ldots & q_{B_{1n}}\\
	 q_{B_{21}} & q_{B_{22}} & \ldots & q_{B_{2n}}\\
	 \vdots		& \vdots	 & \ddots & \vdots\\
	 q_{B_{n1}} & q_{B_{n2}} & \ldots & q_{B_{nn}}\\	 
	\end{bmatrix} \Rightarrow
	q_{B_n}=
	\begin{bmatrix}
		q_{B_{1n}} \\
		q_{B_{2n}} \\
		\vdots \\
		q_{B_{nn}} \\
	\end{bmatrix}	
\]


\subsection{Tranformation auf Jordansche Normalform}
Ein Übertragungssystem 
\[ \dot{x}=A\cdot x +  b \cdot u \]
wird mit der Transformation  
\[ x=V\cdot z \]
in die Jordansche Normalform 
\[ \dot{x}=A_J\cdot x +  b_J \cdot u \]
überführt. Die Matrizen werden folgendermassen Transformiert:
\[\begin{aligned}
	A_J &= V^{-1}\cdot A \cdot V	\\	
	b_J &= V^{-1}\cdot b	\\	
	{c^T}_J &= c^T\cdot V	\\	
	d_J &= d
\end{aligned}\]
Die Transformationsmatrix $V$ ($n\times n$) besteht aus den Eigenvektoren $v_n$ der Matrix $A$:
\[
	V = \begin{bmatrix}
		 v_{11} & v_{12} & \ldots & v_{1n}\\
		 v_{21} & v_{22} & \ldots & v_{2n}\\
		 \vdots		& \vdots	 & \ddots & \vdots\\
		 v_{n1} & v_{n2} & \ldots & v_{nn}\\	 
		\end{bmatrix} =
		\begin{bmatrix}
			v_1 & v_2 & \ldots & v_n \\
		\end{bmatrix}	
\]