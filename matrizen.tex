% coding:utf-8

%----------------------------------------
%FOSAPHY, a LaTeX-Code for a summary of modern control theory
%Copyright (C) 2015, Mario Felder & Michi Fallegger

%This program is free software; you can redistribute it and/or
%modify it under the terms of the GNU General Public License
%as published by the Free Software Foundation; either version 2
%of the License, or (at your option) any later version.

%This program is distributed in the hope that it will be useful,
%but WITHOUT ANY WARRANTY; without even the implied warranty of
%MERCHANTABILITY or FITNESS FOR A PARTICULAR PURPOSE.  See the
%GNU General Public License for more details.
%----------------------------------------

\chapter{Matrizen}

\section{Grundlagen}
\subsection*{$m\times n$ Matrize}
\[
	A = \left[ a_{jk} \right] =
	\begin{bmatrix}
		a_{11} & a_{12} & \ldots & a_{1n} \\
		a_{21} & a_{22} &  & a_{2n} \\
		\vdots &  & \ddots & \vdots \\
		a_{m1} & a_{m2} & \ldots & a_{mn} \\
	\end{bmatrix}
\]
\\
m = Zeilen\\
n = Spalten

\subsection*{Vektoren}
Zeilenvektor:
\[
	\underline{a} = \left[ a_j \right] =
	\begin{bmatrix}
		a_1 & a_2 & \ldots & a_m
	\end{bmatrix}
\]
\\
Spaltenvektor:
\[
	\underline{b} = \left[ b_k \right] =
	\begin{bmatrix}
		b_1 \\ b_2 \\ \vdots \\ b_m \\
	\end{bmatrix}
\]
\subsection*{Transponierte}
\[
	A^T = \left[ a_{kj} \right] =
		\begin{bmatrix}
			a_{11} & a_{21} & \ldots & a_{m1} \\
			a_{12} & a_{22} &  & a_{m2} \\
			\vdots &  & \ddots & \vdots \\
			a_{1n} & a_{2n} & \ldots & a_{mn} \\
		\end{bmatrix}
\]
\\
$A^T = A$ ist eine symmetrische Matrix\\
$A^T = -A$ ist eine schief-symmetrische Matrix

\subsection*{Addition}
Ist nur für Matrizen derselben Dimension $A = [a_{jk}], B=[b_{jk}]$ definiert.
\[
	C = A + B = [a_{jk} + b_{jk}]
\]

\subsection*{Multiplikation}
\textbf{\textit{Definition:}} Das Produkt $C=A\cdot B$ einer $m \times n$-Matrix $A=[a_{jk}]$ und einer $r \times p$-Matrix $B=[b_{jk}]$ ist nur defniert wenn $r=n$ ist.\\
\[
	c_{jk} = \sum_{l=1}^{n} a_{jl} \cdot b_{lk} = a_{j1}b_{1k} + a_{j2}b_{2k} + \ldots + a_{jn}b_{nk}
\]
\\
wobei $j=1,2,\ldots,m$ und $k = 1,2,\ldots,p$.
\[
\begin{aligned}
	C =A \cdot B &= 
	\begin{bmatrix}
		a_{11} & a_{12} & \ldots & a_{1n} \\
		a_{21} & a_{22} &  & a_{2n} \\
		\vdots &  & \ddots & \vdots \\
		a_{m1} & a_{m2} & \ldots & a_{mn} \\
	\end{bmatrix} \cdot
	\begin{bmatrix}
		b_{11} & b_{12} & \ldots & b_{1p} \\
		b_{21} & b_{22} &  & b_{2p} \\
		\vdots &  & \ddots & \vdots \\
		b_{n1} & b_{n2} & \ldots & b_{np} \\
	\end{bmatrix}\\ &= 
	\begin{bmatrix}
		a_{11}b_{11} + \ldots + a_{1n}b_{n1} &  \ldots & a_{11}b_{1n} + \ldots + a_{1n}b_{nn} \\
		a_{21}b_{11} + \ldots + a_{2n}b_{n1} &  & a_{21}b_{1n} + \ldots + a_{2n}b_{nn} \\
		\vdots & \ddots & \vdots \\
		a_{m1}b_{11} + \ldots + a_{mn}b_{n1} & \ldots & a_{m1}b_{1n} + \ldots + a_{mn}b_{nn} \\
	\end{bmatrix}
\end{aligned}
\]
\\
\textbf{Transponierung eines Produkts:}
\[
	\left(A\cdot B\right)^T = B^T \cdot A^T
\]
\subsection*{Spezielle Matrizen}
Eine Quadratische Matrix, deren Elemente oberhalb der Hauptdiagnoale sämtlich Null sind,
heisst \textbf{untere Dreieckmatrix}.\\
Wenn die Elemente unterhalb der Hauptdiagonalen sämtlich Null sind, dann heisst sie \textbf{obere Dreieckmatrix}.
\[
	T_1 =
	\begin{bmatrix}
		t_{11} & 0 		& 0 \\
		t_{21} & t_{22} & 0 \\
		t_{31} & t_{32} & t_{33} \\
	\end{bmatrix} \qquad
	T_2 =
	\begin{bmatrix}
		t_{11} & t_{21} & t_{31} \\
		0	   & t_{22} & t_{32} \\
		0	   & 0      & t_{33} \\
	\end{bmatrix}
\]
\\
Ene quadratische Matrix $A=[a_{jk}]$ deren Elemente unter- und oberhalb der Hauptdiagonalen sämtlich null sind, heisst \textbf{Diagonalmatrix}. ($a_{jk} = 0$ für alle $j \neq k$)
\[
	D = 
	\begin{bmatrix}
		d_{11} & 0      & 0\\
		0	   & d_{22} & 0 \\
		0	   & 0      & d_{33} \\
	\end{bmatrix} \qquad
	S =
	\begin{bmatrix}
		c & 0 & 0\\
		0 & c & 0 \\
		0 & 0 & c \\
	\end{bmatrix}
\]
\\
$S$ ist eine \textbf{Skalarmatrix}, da: $A \cdot S = S \cdot A = cA$\\
Eine Skalarmatrix, deren Elemente der Hauptdiagonale sämtlich 1 sind, heisst \textbf{Einheitsmatrix}.
\[
	I =
	\begin{bmatrix}
		1 & 0 & 0\\
		0 & 1 & 0 \\
		0 & 0 & 1 \\
	\end{bmatrix}
\]

\subsection*{Orthogonal}
Vektor \underline{a} ist zum Vektor \underline{b} orthogonal, wenn das Skalarprodukt $\underline{a}\bullet\underline{b}=0$ ist. Dann ist auch \underline{b} orthogonal zu \underline{a} und wir sprechen daher von zwei orthogonalen Vektoren \underline{a} und \underline{b}.\\
\\
Zwei orthogonale Nichtnullvektoren sind aufeinander senkrecht ($\cos(\alpha)=0,\alpha=\frac{\pi}{2}$).

\subsection*{Determinante}
\[
	D = \det{A}=
	\begin{bmatrix}
	a & b  \\
	c & d  \\
	\end{bmatrix}
	=a \cdot d - b \cdot c
\]
\textbf{Wichtige Eigenschaften:}
\begin{enumerate}
	\item $\det{A} = \det{A^T}$
	\item Sind zwei Zeilen oder Spalten linear abhängig, so ist die Determinante = 0
	\item Vertauscht man zwei Zeilen oder Spalten, so wird die Determinante mit -1 multipliziert
	\item $\det{AB} = \det{BA} = \det{A} \cdot \det{B}$
\end{enumerate}
\textbf{Lineares Gleichungssystem:}
\[
	A \cdot \underline{x} = \underline{b}
\]
hat die Lösungen: $x_1 = \frac{D_1}{D}$, $x_2 = \frac{D_2}{D}$, wobei $D = \det{A}$ und
\[
	D_1 = 
	\begin{bmatrix}
		b_1 & a_{12} \\
		b_2 & a_{22} \\
	\end{bmatrix} \qquad
	D_2 = 
	\begin{bmatrix}
		a_{11} & b_1 \\
		a_{21} & b_2 \\
	\end{bmatrix}
\]

\section{Rang}
\subsection{Rang von Vektoren}
Beschreibt die lineare Abhängigkeit und Unabhängigkeit von Vektoren.
\\
Eine Menge von m Vektoren $a_1,a_2,...,a_m$ ( mit derselben Anzahl von Komponenten ) bildet die folgende lineare Kombination:
\[
	c_1a_1+c_2a_2+...+c_ma_m
\]
Daraus folgt:
\[
	c_1a_1+c_2a_2+...+c_ma_m=0
\]
Falls die einzige Möglichkeit darin besteht, $c=0$ zu setzten um die Gleichung zu erfüllen, sind die Vektoren \underline{linear unabhängig.}
\\
\\
Zwei Vektoren in der Ebene sind \underline{linear abhängig}, wenn sie parallel sind.
\[
	\underline{a}-c\cdot\underline{b}=0=\begin{bmatrix}
	0  \\
	0  \\
	\end{bmatrix}
\]
\\
Drei Vektoren in Anschauungsraum (3D) sind \underline{linear abhängig}, wenn sie in einer Ebene liegen.
\[
	c_1\cdot\underline{a_1}+c_2\cdot\underline{a_2}+c_3\cdot\underline{a_3}=0
\]

\subsection{Rang einer Matrix}
Die maximale Zahl der linear unabhängigen Zeilenvektoren einer Matrix \underline{A} heisst Rang.
Es gilt:
\[
	r= Rang(A)\leq m,n
\]
\\
\textbf{Vorgehen (Horizontal)}:
\begin{enumerate}
 \item Erste Zeile (oder die mit der tiefsten Zahlen) stehen lassen.
 \item Dieser Schritt für alle Zeilen machen:
 \[
 	c_j\cdot a_{11} + a_{j1} = 0 \qquad \rightarrow \qquad
 	c_j \cdot \begin{bmatrix} a_{j1} & a_{j2} & \ldots & a_{jn}  \end{bmatrix} 
 \]
 \item Entstehen in der Matrix horizontale gleiche Vektoren, so sind diese linear abhängig.
\end{enumerate}

\section{Adjunkte}
Ist nur für quadratische Matrizen definiert.
\[
	adj(A) = Cof(A)^T = \tilde{A}^T =
	\begin{bmatrix}
		\tilde{a}_{11} & \tilde{a}_{12} & \ldots & \tilde{a}_{1n}\\
		\tilde{a}_{21} & \tilde{a}_{22} & 		 & \tilde{a}_{2n}\\
		\vdots		   &				& \ddots & \vdots \\
		\tilde{a}_{n1} & \tilde{a}_{n2} & \ldots & \tilde{a}_{nn}\\
	\end{bmatrix}^T
\]
\\
An der Stelle $(j,k)$ steht der Cofaktor $\tilde{a}_{jk}$, welcher sich folgendermassen berechnet:
\[
	\tilde{a}_{jk} = (-1)^{j+k}\cdot M_{jk} = (-1)^{j+k}\cdot \det
	\begin{bmatrix}
		a_{1,1} & \ldots & a_{1,k-1} & a_{1,k+1} & \ldots & a_{1,n} \\
		\vdots  & \ddots & \vdots    & \vdots    & \ddots & \vdots  \\
		a_{j-1,1} & \ldots & a_{j-1,k-1} & a_{j-1,k+1} & \ldots & a_{j-1,n} \\
		a_{j+1,1} & \ldots & a_{j+1,k-1} & a_{j+1,k+1} & \ldots & a_{j+1,n} \\
		\vdots  & \ddots & \vdots    & \vdots    & \ddots & \vdots  \\
		a_{n,1} & \ldots & a_{n,k-1} & a_{n,k+1} & \ldots & a_{n,n} \\
	\end{bmatrix}
\]
\\
\textbf{2$\times$2 Matrix}
\[
	A = 
	\begin{bmatrix}
		a & b \\
		c & d \\
	\end{bmatrix}
\]
\\
Adjunkte:
\[
	adj(A) =
	\begin{bmatrix}
		d & -c \\
		-b & a \\
	\end{bmatrix}^T =
	\begin{bmatrix}
		d & -b \\
		-c & a \\
	\end{bmatrix}
\]

\section{Inverse einer Matrix}
Ist nur für quadratische Matrizen definiert.\\
Die Inverse einer $n \times n$-Matrix ist gegeben durch:
\[
	A^{-1} = \frac{1}{\det{A}} \cdot adj(A) = \frac{1}{\det{A}} \cdot
	\begin{bmatrix}
		\tilde{a}_{11} & \tilde{a}_{21} & \ldots & \tilde{a}_{n1}\\
		\tilde{a}_{12} & \tilde{a}_{22} & 		 & \tilde{a}_{n2}\\
		\vdots		   &				& \ddots & \vdots \\
		\tilde{a}_{1n} & \tilde{a}_{2n} & \ldots & \tilde{a}_{nn}\\
	\end{bmatrix}
\]
\\
Es gilt:
\[
	A \cdot A^{-1} = A^{-1} \cdot A = I
\]

\section{Eigenwerte und Eigenvektoren}
\[
	A \cdot \underline{x} = \lambda \cdot \underline{x}
\]
Derjenige Wert $\lambda$ für welchen die obige Gleichung eine Lösung $\underline{x}\neq 0$ hat heisst der Eigenwert der Matrix \underline{A}. 
\\Die korrespondierende Lösung \underline{x} $\neq 0$ heisst der Eigenvektor der Matrix A.
\\
\[\begin{aligned}
	A \cdot \underline{x }&=\lambda \cdot \underline{x}\\
		\begin{bmatrix}
			a & b  \\
			c & d  \\
		\end{bmatrix}
		\cdot	
		\begin{bmatrix}
				x_1 \\
				x_2 \\
		\end{bmatrix}
		&=
		\lambda
		\begin{bmatrix}
				x_1 \\
				x_2 \\
		\end{bmatrix}
\end{aligned}
\]
\\
Homogenes, lineares Gleichungssystem
\[
	(A-\lambda\cdot I) \underline{x} = \underline{0}
\]
\\
\textbf{Lösung nach Cramer: Eigenwert bestimmen}
\[
		D(\lambda)=\det(A-\lambda I)= 0	\\	\lambda I = \begin{bmatrix}
			\lambda & 0  \\
			0 & \lambda  \\
		\end{bmatrix}
\]
\\
\[
		\lambda_{1,2}=\frac{-b\pm \sqrt{b^2-4ac}}{2a}	\\	a	\lambda^2+b	\lambda+c=0
\]
\\
Eigenvektor
\[
		\underline{x}= A - \lambda \cdot I 
\]