% coding:utf-8

%----------------------------------------
%FOSAPHY, a LaTeX-Code for a summary of basic physics
%Copyright (C) 2013, Mario Felder

%This program is free software; you can redistribute it and/or
%modify it under the terms of the GNU General Public License
%as published by the Free Software Foundation; either version 2
%of the License, or (at your option) any later version.

%This program is distributed in the hope that it will be useful,
%but WITHOUT ANY WARRANTY; without even the implied warranty of
%MERCHANTABILITY or FITNESS FOR A PARTICULAR PURPOSE.  See the
%GNU General Public License for more details.
%----------------------------------------

\chapter{Regelung im Zustandsraum}

\section{Grundlagen von Matrizen}
\textbf{Matrize} \\
\\
$A = 
\begin{bmatrix}
a_{11} & a_{12} & a_{13} & a_{1n} \\
a_{21} & a_{22}& a_{23} & a_{2n} \\
a_{31} & a_{32} & a_{33} & a_{3n} \\
a_{m1} & a_{m2} & a_{m3} & a_{mn} \\
\end{bmatrix}
$   
\\
\\
A = [Spalten,Zeilen]
\\
\\
\textbf{Transponierte} \\
\\
$A = 
\begin{bmatrix}
a_{11} & a_{12} & a_{13} & a_{1n} \\
a_{21} & a_{22}& a_{23} & a_{2n} \\
a_{31} & a_{32} & a_{33} & a_{3n} \\
a_{m1} & a_{m2} & a_{m3} & a_{mn} \\
\end{bmatrix}^T
$  
\
=
$\begin{bmatrix}
a_{11} & a_{21} & a_{31} & a_{n1} \\
a_{12} & a_{22} & a_{32} & a_{n2} \\
a_{13} & a_{23} & a_{33} & a_{n3} \\
a_{1n} & a_{2m} & a_{3m} & a_{nm} \\
\end{bmatrix}
$  
\\
\\
\\
\textbf{Multiplikation} \\
\underline{A}*\underline{B}=$\begin{bmatrix}
a_{11} & a_{21}  \\
a_{12} & a_{22}  \\
\end{bmatrix}$
*
$\begin{bmatrix}
b_{11} & b_{21}  \\
b_{12} & b_{22}  \\
\end{bmatrix}$
=
$\begin{bmatrix}
a_{11}*b_{11}+a_{21}*b_{12} & a_{11}*b_{21}+a_{21}*b_{22}  \\
a_{12}*b_{11}+a_{22}*b_{12} & a_{12}*b_{21}+a_{22}*b_{22}  \\
\end{bmatrix}$
\\
\\
\textbf{Orthogonal} \\
Vektor \underline{a} ist zum Vektor \underline{b} orthogonal, wenn das Skalarprodukt \underline{a}$*$\underline{b}$=0$ ist. Dann ist auch \underline{b} orthogonal zu \underline{a} und wir sprechen daher von zwei orthogonalen Vektoren \underline{a} und \underline{b}.\\
\\
Zwei orthogonale Nichtnullvektoren sind aufeinander senkrecht ($\cos(\alpha)=0,\alpha=\frac{\pi}{2}$).
\\
\\
\textbf{Determinante}
\[
	det(A)=
	\begin{bmatrix}
	a & b  \\
	c & d  \\
	\end{bmatrix}
	=a \cdot d - b \cdot c
\]
\\
\section{Rang}
\subsection{Rang von Vektoren}
Beschreibt die lineare Abhängigkeit und Unabhängigkeit von Vektoren.
\\
Eine Menge von m Vektoren $a_1,a_2,...,a_n$ ( mit derselben Anzahl von Komponenten ) bildet die folgende lineare Kombination:
\[
	c_1a_1+c_2a_2+...+c_ma_m
\]
Daraus folgt:
\[
	c_1a_1+c_2a_2+...+c_ma_m=0
\]
Falls die einzige Möglichkeit darin besteht, $c=$ um die Gleichung zu erfüllen, sind die Vektoren \underline{linear unabhängig.}
\\
\\
Zwei Vektoren in der Ebene sind \underline{linear abhängig}, wenn sie parallel sind.
\[
	\underline{a}-c*\underline{b}=0=\begin{bmatrix}
	0  \\
	0  \\
	\end{bmatrix}
\]
\\
Drei Vektoren in Anschauungsraum (3D) sind \underline{linear abhängig}, wenn sie in einer Ebene liegen.
\[
	c_1\cdot\underline{a}+c_2\cdot\underline{b}+c_3\cdot\underline{c}=0
\]
\subsection{Rang einer Matrix}
Die maximale Zahl der linear unabhängigen Zeilenvektoren einer Matrix \underline{A} heisst Rang.
Es gilt:
\[
	r= Rang(A)\leq m,n
\]
\[
\begin{bmatrix}
	a_{11} & a_{12} & a_{13} \\
	a_{21} & a_{22}& a_{23}  \\
	a_{31} & a_{32} & a_{33}  \\
	\end{bmatrix}
\	
	=Rang
	\begin{pmatrix}
		a_{11} & a_{12} & a_{13} \\
		a_{21} & a_{22}& a_{23}  \\
		a_{31} & a_{32} & a_{33}  \\
		\end{pmatrix}
\]
\\
\textbf{Vorgehen (Horizontal)}:\\
-1. Erste Zeile (oder die mit der tiefsten Zahlen) stehen lassen.\\
-2. Dieser Schritt für alle Zeilen machen:
\[
	c_1\cdot a_{11} + a_{21} = 0\\
	c\begin{bmatrix} a_{21} & a_{22}& a_{23}  \\  \end{bmatrix} 
\]
-3. Entstehen in der Matrix horizontale gleiche Vektoren, so sind diese linear abhängig.
\\
\section{Eigenwerte und Eigenvektoren}
\subsection{Eigenwerte}
\[
	A\underline{v} = \lambda \underline{v}
\]
Derjenige Wert $\lambda$ für welchen die obige Gleichung eine Lösung $x\neq 0$ hat heisst der Eigenwert der Matrix \underline{A}. 
\\Die korrespondierende Lösung \underline{x} $\neq 0$ heisst der Eigenvektor der Matrix A.
\\
\[
	A \cdot \underline{x }=\lambda \cdot \underline{x}
\]
\[
		\begin{bmatrix}
			a & b  \\
			c & d  \\
		\end{bmatrix}
		\cdot	
		\begin{bmatrix}
				x_1 \\
				x_2 \\
		\end{bmatrix}
		=
		\lambda
		\begin{bmatrix}
				x_1 \\
				x_2 \\
		\end{bmatrix}
\]
\\
Homogenes, lineares Gleichungssystem
\[
	(A-\lambda\cdot I) \underline{x} = \underline{0}
\]
\\
\textbf{Lösung nach Cramer: Eigenwert bestimmen}
\[
		D(\lambda)=det(A-\lambda I)= 0	\\	\lambda I = \begin{bmatrix}
			\lambda & 0  \\
			0 & \lambda  \\
		\end{bmatrix}
\]
\\
\[
		\lambda_{1,2}=\frac{-b\pm \sqrt{b^2-4ac}}{2a}	\\	a	\lambda^2+b	\lambda+c=0
\]
\\
Eigenvektor
\[
		\underline{v}= A - \lambda \cdot I 
\]
\\
\\
\section{Zustandsvariabel}
Es gibt allgemeine Untersuchungen über das Systemverhalten, beispielsweise über die Steuerbarkeit und Beobachtbarkeit von Systemen, die sich nur im Zeitbereich durchführen lassen.\\
Aus diesem Grund ist es angebracht im Zeitbereich zu bleiben. Es ist zweckmässig, die auftretende Differentialgleichung durch Einführen von Zwischengrössen in Systeme von Differentialgleichungen erster Ordnung zu verwandeln.
\[
	a_n\cdot y^n+a_{n-1} \cdot y^{n-1}+...+a_2\ddot{y} +a_1\dot{y}+ a_0y=b_0u 
	\\ a_n\neq 0
\]
Dann kann man als Zwischengrössen $x_1,x_2,...,x_n$ die Ausgangsgrösse $y$ und ihre Ableitungen nehmen:
\[
	x_1=y,	\\	x_2=\dot{y},	\\ x_3=\ddot{y},...,	\\x_{n-1}=y^{n-2}	\\	x_n=y^{n-1}
\]
Aus der Definition folgen die einfachen Differentialgleichungen.:
\[
	\dot{x}_1=x_2,	\\	\dot{x}_2=x_3	\\,...,	\\ 	\dot{x}_{n-1}=x_n
\]
