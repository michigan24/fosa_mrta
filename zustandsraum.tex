% coding:utf-8

%----------------------------------------
%FOSAPHY, a LaTeX-Code for a summary of basic physics
%Copyright (C) 2013, Mario Felder

%This program is free software; you can redistribute it and/or
%modify it under the terms of the GNU General Public License
%as published by the Free Software Foundation; either version 2
%of the License, or (at your option) any later version.

%This program is distributed in the hope that it will be useful,
%but WITHOUT ANY WARRANTY; without even the implied warranty of
%MERCHANTABILITY or FITNESS FOR A PARTICULAR PURPOSE.  See the
%GNU General Public License for more details.
%----------------------------------------

\chapter{Zustandsraum}
\section{Zustandsvariabel}
Es gibt allgemeine Untersuchungen über das Systemverhalten, beispielsweise über die Steuerbarkeit und Beobachtbarkeit von Systemen, die sich nur im Zeitbereich durchführen lassen.\\
Aus diesem Grund ist es angebracht im Zeitbereich zu bleiben. Es ist zweckmässig, die auftretende Differentialgleichung durch Einführen von Zwischengrössen in Systeme von Differentialgleichungen erster Ordnung zu verwandeln.
\[
	a_n\cdot y^n+a_{n-1} \cdot y^{n-1}+...+a_2\ddot{y} +a_1\dot{y}+ a_0y=b_0u 
	\\ a_n\neq 0
\]
Dann kann man als Zwischengrössen $x_1,x_2,...,x_n$ die Ausgangsgrösse $y$ und ihre Ableitungen nehmen:
\[
	x_1=y,	\\	x_2=\dot{y},	\\ x_3=\ddot{y},...,	\\x_{n-1}=y^{n-2}	\\	x_n=y^{n-1}
\]
Aus der Definition folgen die einfachen Differentialgleichungen.:
\[
	\dot{x}_1=x_2,	\\	\dot{x}_2=x_3	\\,...,	\\ 	\dot{x}_{n-1}=x_n
\]
\section{Regelungsnormalform der Zustandsgleichung}
\[
	Y(s) = G(s) \cdot U(s)
\]
\[
	G(s)= \frac{Z(s)}{N(s)} = \frac{b_0+b_1s+b_2s^2+...+b_ns^n}{a_0+a_1s+a_2s^2+...+a_ns^n}
\]
\subsection{Regelungsnormalform}
\[
	\dot x=
	\underbrace{
		\begin{bmatrix}
			0 &	1 & 0 & \ldots & 0\\
			0 & 0 & 1 & \ldots & 0\\
			\vdots & \vdots & \vdots & \ddots & \vdots \\
			0 & 0 & 0 & \ldots & 1\\
			-\frac{a_0}{a_n} &-\frac{a_1}{a_n} & -\frac{a_2}{a_n} & \ldots &-\frac{a_{n-1}}{a_n}\\	
		\end{bmatrix}
	}_{\textbf{A}}
	\cdot x +
	\underbrace{
		\begin{bmatrix}
			0 \\
			0 \\
			\vdots \\
			0 \\
			\frac{1}{a_n} \\	
		\end{bmatrix}
	}_{\textbf{b}}
	\cdot u	
\]

\[
	y=
	\underbrace{
			\begin{bmatrix}
				b_0-a_0\cdot\frac{b_n}{a_n} & b_1-a_1\cdot\frac{b_n}{a_n} & \ldots & b_{n-1}-a_{n-1}\cdot\frac{b_n}{a_n} &\\
			\end{bmatrix}
	}_{\textbf{$c^T$}}
	\cdot x  +
	\underbrace{
		\left[ \frac{b_n}{a_n} \right] 
	}_{\textbf{d}}
	\cdot u
\]

\subsection{Beobachtungsnormalform}
\[
	\dot x=
	\underbrace{
		\begin{bmatrix}
			0 &	0 & 0 & \ldots & -\frac{a_0}{a_n}\\
			1 & 0 & 0 & \ldots & -\frac{a_1}{a_n}\\
			0 & 1 & 0 & \ldots & -\frac{a_2}{a_n}\\
			\vdots & \vdots & \vdots & \ddots & \vdots \\
			0 & 0 & \ldots & 1 &-\frac{a_{n-1}}{a_n}\\	
		\end{bmatrix}
	}_{\textbf{A}}
	\cdot x +
	\underbrace{
		\begin{bmatrix}
			b_0-b_n\frac{a_0}{a_n} \\
			b_1-b_n\frac{a_1}{a_n} \\
			b_2-b_n\frac{a_2}{a_n}  \\
			\vdots\\
			b_{n-1}-b_n\frac{a_{n-1}}{a_n}\\	
		\end{bmatrix}
	}_{\textbf{b}}
	\cdot u	
\]
\[
	y=
	\underbrace{
			\begin{bmatrix}
				0 & 0 & \ldots & 0 & \frac{1}{a_n}\\
			\end{bmatrix}
	}_{\textbf{$c^T$}}
	\cdot x  +
	\underbrace{
		\left[ \frac{b_n}{a_n} \right] 
	}_{\textbf{d}}
	\cdot u
\]
\subsection{Jordanische Normalform}
\begin{itemize}
      \item Bevorzugte Verwendung, wenn Pole vom System bekannt sind
      \item System ist vollständig entkoppelt, wenn alle Pole reell und einfach vorkommen
      \item A ist Diagonalmatrix mit $\lambda_i$: Pole
\end{itemize}
Es gilt:
\[
	Y(s) = \left( \sum_{i=1}^{n} \frac{r_i}{s-\lambda_i} + r_0 \right) \cdot U(s)
\]
Dabei ist $r_0 \neq 0$ wenn Zähler und Nenner von $G(s)$ den gleichen Grad haben.
\[
	\dot x=
	\underbrace{
		\begin{bmatrix}
			\lambda_1 &	0 & 0 & \ldots & 0\\
			0 & \lambda_2 & 0 & \ldots & 0\\
			0 & 0 & \lambda_2 & \ldots & 0\\
			\vdots & \vdots & \vdots & \ddots & \vdots \\
			0 & 0 & 0 & \ldots & \lambda_n\\	
		\end{bmatrix}
	}_{\textbf{A}}
	\cdot x +
	\underbrace{
		\begin{bmatrix}
			1 \\
			1 \\
			1 \\
			\vdots \\
			1\\	
		\end{bmatrix}
	}_{\textbf{b}}
	\cdot u	
\]

\[
	y=
	\underbrace{
			\begin{bmatrix}
				r_1 & r_2 & \ldots & r_n\\
			\end{bmatrix}
	}_{\textbf{$c^T$}}
	\cdot x  +
	\underbrace{
		\left[ r_0 \right] 
	}_{\textbf{d}}
	\cdot u
\]

\subsubsection{Mehrfachpole}
Wenn $\lambda_1$ ein $m$-facher Pol ist gilt:
\[
	Y(s) = \left( \frac{r_1}{s-\lambda_1} + \ldots + \frac{r_m}{\left(s-\lambda_m\right)^m} + \sum_{i=m+1}^{n} \frac{r_i}{s-\lambda_i} + r_0\right) \cdot U(s)
\]
Für die Matrix bedeutet dies:
\[
	\dot x=
	\underbrace{
		\begin{bmatrix}
			\lambda_1 &	 		  &  		& 	& & & \\
			1 		  & \lambda_1 &  		&  	& & &\\
			 		  & \ddots 	  & \ddots 	& 	& & &\\
				  	  & 	 	  & 1 		& \lambda_1 & & & \\
			 	      &  		  &  		&  	& \lambda_{m+1} & &\\	
			 	      &  		  &  		&  	& 				& \ddots &\\
			 	      &  		  &  		&  	& 				&  & \lambda_n\\
		\end{bmatrix}
	}_{\textbf{A}}
	\cdot x +
	\underbrace{
		\begin{bmatrix}
			1 \\
			0 \\
			\vdots \\
			0 \\
			1\\
			\vdots\\
			1	
		\end{bmatrix}
	}_{\textbf{b}}
	\cdot u	
\]

\[
	y=
	\underbrace{
			\begin{bmatrix}
				r_1 & \ldots & r_m & r_{m+1} & \ldots & r_n\\
			\end{bmatrix}
	}_{\textbf{$c^T$}}
	\cdot x  +
	\underbrace{
		\left[ r_0 \right] 
	}_{\textbf{d}}
	\cdot u
\]

\section{Steuerbarkeit und Beobachtbarkeit}
\subsection{Steuerbarkeit}
Das System heisst steuerbar, wenn sein Zustandspunkt \underline{x} durch geeignete Wahl des Steuervektors u in endlicher Zeit aus einem beliebigen Anfangszustand $x_0$ in den Endzustnad 0 bewegt werden kann (Steuerbar wenn die Vektoren linear unabhängig sind; $Rang(Q_s) = n$ oder $det(Q_s)\neq 0$)
\[
	Q_s = 
	\begin{bmatrix}
			b	&	A\cdot b	& .. & A^{n-1} \cdot b\\
	\end{bmatrix}
	= \left[ n \cdot n\right] 
\]

\subsection{Beobachtbarkeit}
Das System heisst beobachtbar, wenn dem bekannten $u(t)$ aus der Messung $y(t)$ über eine endliche Zeitspanne der Anfangszustand $x(t)$ eindeutig ermittelt werden kann, ganz gleich wo dieser liegt.\\
\\
Das System ist beobachtbar, wenn die Beobachtungsmatrix $Q_B$ regulär ist (nxn Matrix) ($det(Q_B)\neq 0$).
\[
	Q_B=
	\begin{bmatrix}
		c^T\\
		c^T \cdot A\\
		...\\
		c^T\cdot A^{n-1}\\
	\end{bmatrix}
\]
\section{Transformation}
\subsection{Transformation in Regelungsnormalform (Steuernormalform)}
Eine Übertragungssystem $\dot{x}=A\cdot x +  b \cdot u$ wird mit der Transformation $z=T_R\cdot x$ in die Regelungsnormalform  $\dot{x_R}=A_R\cdot x_R +  b_R \cdot u$ überführt.\\
\\
$q_{s_n}^T$ ist die letzte Zeile von der inversen Steuerbarkeitsmatrix $Q_s^{-1}$.
\[
	A_R = T_R\cdot A \cdot T_R^{-1}	\\	b_R = T_R\cdot b	\\	c_R^T = c^T\cdot T_R^{-1}	\\	d_R = d
\]
\[
	T_R=
	\begin{bmatrix}
		q_{s_n}^T \\
		q_{s_n}^T \cdot A \\
		.. \\
		q_{s_n}^T \cdot A^{n-1}\\	
	\end{bmatrix}	\\
	Q_s^{-1} =
	\begin{bmatrix}
			.. &	.. & .. & .. & ..\\
			.. &	.. & .. & .. & ..\\
			.. &	.. & .. & .. & ..\\
			 [x &	x & x & x & x]  \\	 
	\end{bmatrix}
\]

\subsection{Transformation auf Beobachtungsnormalform}
Eine Übertragungssystem $\dot{x}=A\cdot x +  b \cdot u$ wird mit der Transformation  $z=T_B\cdot x$ in die Beobachtungsnormalform $\dot{x_B}=A_R\cdot x_B +  b_B \cdot u$ überführt.\\
\\
$q_{B_n}$ ist die letzte Spalte von der inversen Beobachtungsmatrix $Q_B^{-1}$.
\\
\[
	A_B = T_B\cdot A \cdot T_B^{-1}	\\	b_B = T_B\cdot b	\\	c_B^T = c^T\cdot T_B^{-1}	\\	d_B = d
\]
\\
\[
	T_B=
	\begin{bmatrix}
		q_{B_n} & A\cdot q_{B_n} &  ..& A^{n-1}\cdot q_{B_n} & \\
	\end{bmatrix}	\\
	Q_B^{-1} =
	\begin{bmatrix}
			 .. & .. & x\\
			 .. & .. & x\\
			 .. & .. & x\\	 
	\end{bmatrix}
\]

\section{Reglersynthese im Zustandsraum}
\begin{center}
	\includegraphics[scale = 0.4]{images/zustandsregler.png}
\end{center}
\[
	A_{CL}= A-b\cdot r^T	\\ A_R=T_R\cdot A \cdot T_R^{-1}
\]
\[
	A_{R,CL}= A_R-b_R\cdot r_R^T	\\	r_R^T=
		\begin{bmatrix}
				r_{1,R}	&	r_{2,R}	& r_{3,R} &r_{n,R}\\
		\end{bmatrix}
\]
Daraus folgt:
\[
		A_{Cl,R}=
		\begin{bmatrix}
			0 &	1 & 0 & .. & 0\\
			0 & 0 & 1 & .. & 0\\
			.. & .. & .. &.. & .. \\
			0 & 0 & 0 & .. & 1\\
				\underbrace{
					-\frac{a_0}{a_n}-r_{1,R} 
				}_{\textbf{$-c^{n-1}$}}
		 &-\frac{a_1}{a_n}-r_{2,R} & -\frac{a_2}{a_n}-r_{3,R} &.. &-\frac{a_{n-1}}{a_n}-r_{n,R}\\	
		\end{bmatrix}
\]
Diese Matrix kann gerade in das Istpolynom überführt werden.
\\
Istpolynom:
\[
	p_{Cl,r}(s)=s^n+
	\underbrace{(a_{n-1}+r_{n,R})	}_{\textbf{$c^{n-1}$}}
	\cdot s^{n-1}+...+(a_{1}+r_{2,R})a_1\cdot s +(a_0+r_{1,r})
\]
Das Sollpolynom ist durch die Nullstellen vorgegeben:
\[
	p(s)=s^n+p_{n-1}\cdot s^{n-1}+...+p_1\cdot s +p_0
\]
Koeffizientenvergleich ergibt nun:
\[
	r_{1,R}=p_0-a_0
\]
\[
	r_{2,R}=p_1-a_1
\]
\[
	r_{n,R}=p_{n-1}-a_{n-1}
\]
\\
\\
Durch die Transformation in RNF der letzten Zeile von $Q_s^{-1}$ (T) lässt sich die Matrix umwandeln.
\[
	r^T=r_R^T \cdot T
\]


\subsection{Vorfilter / Vorverstärker}
Der Vorfilter/Vorverstärker gewährleistet, dass im stationärem Zustand y mit dem gewünschtem, konstantem Vektor w übereinstimmt.
\[
	 	v=[c^T(b\cdot r^T-A)^{-1}\cdot b]^{-1}
\]
\\
\subsection{Beobachter}
Beobachter ist ein Nachbau des System für den Rechner. Er beinhaltet alle Punkte des realen System. Dafür werden keine Sensoren benötigt. Die Variablen werden geschätzt.\\
\\
Das $h$ muss derart bestimmt werden, dass $eigW(A-h_c^T)<0$ sind.\\
\\
Falls das System in BNF vorliegt, gilt:
\[
	A_B=T_B^{-1}\cdot A \cdot T_b	\\	c_B^T=c^T\cdot T_B = [0 \ 0 \ .. \  1]	\\	b_B=T_B^{-1}\cdot b
\]
\[
	A_B= \begin{bmatrix}
				0 &	0 & 0 & .. & -\frac{a_0}{a_n}\\
				1 & 0 & 0 & .. & -\frac{a_1}{a_n}\\
				0 & 1 & 0 & .. & -\frac{a_2}{a_n}\\
				.. & .. & .. &.. & .. \\
				0 & 0 & .. & 1 &-\frac{a_{n-1}}{a_n}\\	
			\end{bmatrix}
\]
Folgende Gleichung wird für die Bestimmung des Istpolynoms benötigt. Da $a_n=1$, kann folgende Vereinfachung gemacht werden:
\[
	\dot{e}_{x,B}=(A_B-h_B\cdot c_B^T)\cdot e_{x,B}=
	\begin{bmatrix}
					0 &	0 & 0 & .. & -a_0-h_{B,1}\\
					1 & 0 & 0 & .. & -a_1-h_{B,2}\\
					0 & 1 & 0 & .. & -a_2-h_{B,3}\\
					.. & .. & .. &.. & .. \\
					0 & 0 & .. & 1 &-a_{n-1}-h_{B,n}\\	
				\end{bmatrix}
\]

Istpolynom:
\[
	u(s)=s^n+(a_{n-1}+h_{B,n})s^{n-1}+...+(a_1+h_{B,2})s+(a_0+h_{B,1})
\]
Sollpolynom:
\[
	p(s)=s^n+p_{n-1}s^{n-1}+...+p_1 s+p_0
\]
Aus diesen beiden Gleichungen kann via Koeffizientenvergleich die Matrix $h_B$ bestimmt werden.
\[
	h_{B,1}=p_0-a_0	\\	h_{B,2}=p1-a1	\\	etc.	
\]
\[
	h_B=p-a
\]