% coding:utf-8

%----------------------------------------
%FOSAPHY, a LaTeX-Code for a summary of modern control theory
%Copyright (C) 2015, Mario Felder & Michi Fallegger

%This program is free software; you can redistribute it and/or
%modify it under the terms of the GNU General Public License
%as published by the Free Software Foundation; either version 2
%of the License, or (at your option) any later version.

%This program is distributed in the hope that it will be useful,
%but WITHOUT ANY WARRANTY; without even the implied warranty of
%MERCHANTABILITY or FITNESS FOR A PARTICULAR PURPOSE.  See the
%GNU General Public License for more details.
%----------------------------------------

\section{PI - Zustandsregler}
Der PI - Zustandsregler kompensiert eine permanente Last(Störung)
\[
	\dot x_{PI}
	=
	\begin{bmatrix}
	\dot{x} \\
	\dot{x_I} \\
	\end{bmatrix}
	=
	\setlength{\tabcolsep}{10pt} % Abstände zwischen den Spalten 
	\renewcommand*{\arraystretch}{1.3} % Abstände in Tabellen von den Zeilen 
	\setlength{\dashlinegap}{2pt}
	\underbrace{\left[\begin{array}{c @{$ ¦ $}c}
	A & 0  \\ \hdashline
	c^T & 0   \\
	\end{array}\right]}_{A_{PI}}
	\cdot
	\begin{bmatrix}
	x \\
	x_I \\
	\end{bmatrix}
	+
	\underbrace{\begin{bmatrix}
	b \\
	0 \\
	\end{bmatrix}}_{b_{PI}}
	\cdot
	u
\]
\[
	y =
	\underbrace{\left[\begin{array}{c @{$ ¦$} c}
		c^T & 0
		\end{array}\right]}_{c^T_{PI}}
	\cdot
	\begin{bmatrix}
	x \\
	x_I \\
	\end{bmatrix}
\]

\begin{description}
	\item[Entwurfsschritte]~\par
	\begin{itemize}
		\item Bildung des erwateten Systems ($ A_{PI}, b_{PI}, c^T_{PI} $)
		\item Polvorgabe oder LQR für die Berechnung von $ r^T_{PI} $ 
		\item Der letzte Wert von $r^T_{PI} $ ist $-K_I$
		\item Vorgabe von Kp experimentell oder gemäss separatem Vorgehen
		\item $r^T$ kann aus den ersten $n$-Elementen von $r^T_{PI} $ berechnet werden
		\item Kontrolle der Lage der Eigenwerte
		\begin{itemize}
			\item $det(\lambda I_{(n+1)}-(A_{PI}-b_{PI}\cdot r^T_{PI}))=0 $
			\item $eig(A_{PI}-b_{PI}\cdot r^T_{PI})	$
		\end{itemize} 	
	\end{itemize}
\end{description}

\begin{description}
	\item [Mögliche Bestimmung von $K_p$]
	\[
		K_p=\alpha \cdot [-c^T\cdot A^{-1}\cdot b]^{-1}
	\]
	$\alpha $ ist frei wählbar und stellt den Bezug zur Stellgrösse $ u(0) $ bei schrittförmiger Änderung der Eingangsgrösse dar: 
	\[ u(0)\leqslant \alpha\cdot u(\infty)
	\]
\end{description}