% coding:utf-8

%----------------------------------------
%FOSAPHY, a LaTeX-Code for a summary of modern control theory
%Copyright (C) 2015, Mario Felder & Michi Fallegger

%This program is free software; you can redistribute it and/or
%modify it under the terms of the GNU General Public License
%as published by the Free Software Foundation; either version 2
%of the License, or (at your option) any later version.

%This program is distributed in the hope that it will be useful,
%but WITHOUT ANY WARRANTY; without even the implied warranty of
%MERCHANTABILITY or FITNESS FOR A PARTICULAR PURPOSE.  See the
%GNU General Public License for more details.
%----------------------------------------

\section{Zustandsvariabel}
Es gibt allgemeine Untersuchungen über das Systemverhalten, beispielsweise über die Steuerbarkeit und Beobachtbarkeit von Systemen, die sich nur im Zeitbereich durchführen lassen.\\
Aus diesem Grund ist es angebracht im Zeitbereich zu bleiben. Es ist zweckmässig, die auftretende Differentialgleichung durch Einführen von Zwischengrössen in Systeme von Differentialgleichungen erster Ordnung zu verwandeln.
\[
	a_n\cdot y^n+a_{n-1} \cdot y^{n-1}+...+a_2\ddot{y} +a_1\dot{y}+ a_0y=b_0u 
	\\ a_n\neq 0
\]
Dann kann man als Zwischengrössen $x_1,x_2,...,x_n$ die Ausgangsgrösse $y$ und ihre Ableitungen nehmen:
\[
	x_1=y,	\\	x_2=\dot{y},	\\ x_3=\ddot{y},...,	\\x_{n-1}=y^{n-2}	\\	x_n=y^{n-1}
\]
Aus der Definition folgen die einfachen Differentialgleichungen.:
\[
	\dot{x}_1=x_2,	\\	\dot{x}_2=x_3	\\,...,	\\ 	\dot{x}_{n-1}=x_n
\]

\subsection{Laplace-Tranformation auf Zustandsgleichung}
Zeitbereich:
\[
	x(t) = \int_{0}^{t}(Ax(\tau)+Bu(\tau))\di\tau + x(0)
\]

Zustandsdifferentialgleichung:
\[
	\dot{x} = A \cdot x + b \cdot u
\]
Laplace-Tranformation:
\[\begin{aligned}
	sX(s)-x_0 &= A \cdot X(s) + b \cdot U(s) \\
	X(s) &= (sI-A)^{-1} \cdot b \cdot U(s) + (sI-A)^{-1}x_0
\end{aligned}\]
Die inverse Matrix zu $sI-A$ existiert nur wenn $\det\left(sI-A\right) \neq 0$:
\[
	\det\left(sI-A\right) =\begin{bmatrix}
		s-a_{11} & -a_{12} & \ldots & -a_{1n}\\
		-a_{21}	 & s-a_{22}& \ldots & -a_{2n} \\
		\vdots	 & 		   & \ddots & \\
		-a_{n1}	 & -a_{n2} & \ldots & s-a_{nn}\\
	\end{bmatrix}
\]