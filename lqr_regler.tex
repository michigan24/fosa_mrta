% coding:utf-8

%----------------------------------------
%FOSAPHY, a LaTeX-Code for a summary of modern control theory
%Copyright (C) 2015, Mario Felder & Michi Fallegger

%This program is free software; you can redistribute it and/or
%modify it under the terms of the GNU General Public License
%as published by the Free Software Foundation; either version 2
%of the License, or (at your option) any later version.

%This program is distributed in the hope that it will be useful,
%but WITHOUT ANY WARRANTY; without even the implied warranty of
%MERCHANTABILITY or FITNESS FOR A PARTICULAR PURPOSE.  See the
%GNU General Public License for more details.
%----------------------------------------

\section{LQ - Regler}
Der Reglerentwurf kann mittels dem Gütekriterium bestimmt werden. Der Regler wird optimal, wenn die Funktion $ I(r^T) $ minimal wird.
\[
I(r^T)=\int {x^T\cdot Q\cdot x+u^T\cdot R \cdot dt}
\]
\subsection{Liapanov - Gleichung}
\[
S\cdot A_{CL} + A^T_{CL}\cdot S = -Q_r
\]
\subsection{Riccati - Gleichung}
Es gibt genau eine positiv-definite Lösung für $ S $, wenn die Strecke steuerbar ist.
\[
-A^T\cdot S-S\cdot A+S\cdot b\cdot R^{-1} \cdot b^T \cdot S-Q=0	
\]
\paragraph{$ r^T $ - Regelung}	
	\begin{itemize}
		\item $ r=S\cdot b\cdot R^{-1} $
		\item $ r^T=R^{-1}\cdot b^T\cdot S $
	\end{itemize}

\paragraph{LQ - Reglerentwurf}
	\begin{enumerate}
		\item Vorgabe von Q und R
		\begin{itemize}
			\item $ Q=I $ und $ Q=I $
			\item $ Q=c\cdot c^T $ und $ R= \dfrac{1}{100} $
		\end{itemize}
		\item Lösungsmatrix S aus der Riccatigleichung bestimmen
		\item Regelvektor $ r^T $ berechnen
		\item Überprüfung der Eigenwerte
	\end{enumerate}

\begin{description}
	\item[Vorteile LQ - Regler]~\par
	\begin{itemize}
		\item Vorgabe Q, R statt Pole
		\item Phasenreserve $\varphi_r  \geqslant 60 $\textdegree 
		\item Amplitudenreserve $ A_r \geqslant 2 $		
	\end{itemize}
\end{description}
