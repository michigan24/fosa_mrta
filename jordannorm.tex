% coding:utf-8

%----------------------------------------
%FOSAPHY, a LaTeX-Code for a summary of modern control theory
%Copyright (C) 2015, Mario Felder & Michi Fallegger

%This program is free software; you can redistribute it and/or
%modify it under the terms of the GNU General Public License
%as published by the Free Software Foundation; either version 2
%of the License, or (at your option) any later version.

%This program is distributed in the hope that it will be useful,
%but WITHOUT ANY WARRANTY; without even the implied warranty of
%MERCHANTABILITY or FITNESS FOR A PARTICULAR PURPOSE.  See the
%GNU General Public License for more details.
%----------------------------------------

\subsection{Jordanische Normalform}
\begin{itemize}
      \item Bevorzugte Verwendung, wenn Pole vom System bekannt sind
      \item System ist vollständig entkoppelt, wenn alle Pole reell und einfach vorkommen
      \item A ist Diagonalmatrix mit $\lambda_i$: Pole
\end{itemize}
Es gilt:
\[
	Y(s) = \left( \sum_{i=1}^{n} \frac{r_i}{s-\lambda_i} + r_0 \right) \cdot U(s)
\]
Dabei ist $r_0 \neq 0$ wenn Zähler und Nenner von $G(s)$ den gleichen Grad haben.
\[
	\dot x=
	\underbrace{
		\begin{bmatrix}
			\lambda_1 &	0 & 0 & \ldots & 0\\
			0 & \lambda_2 & 0 & \ldots & 0\\
			0 & 0 & \lambda_2 & \ldots & 0\\
			\vdots & \vdots & \vdots & \ddots & \vdots \\
			0 & 0 & 0 & \ldots & \lambda_n\\	
		\end{bmatrix}
	}_{\textbf{A}}
	\cdot x +
	\underbrace{
		\begin{bmatrix}
			1 \\
			1 \\
			1 \\
			\vdots \\
			1\\	
		\end{bmatrix}
	}_{\textbf{b}}
	\cdot u	
\]

\[
	y=
	\underbrace{
			\begin{bmatrix}
				r_1 & r_2 & \ldots & r_n\\
			\end{bmatrix}
	}_{\textbf{$c^T$}}
	\cdot x  +
	\underbrace{
		\left[ r_0 \right] 
	}_{\textbf{d}}
	\cdot u
\]

\subsubsection{Mehrfachpole}
Wenn $\lambda_1$ ein $m$-facher Pol ist gilt:
\[
	Y(s) = \left( \frac{r_1}{s-\lambda_1} + \ldots + \frac{r_m}{\left(s-\lambda_m\right)^m} + \sum_{i=m+1}^{n} \frac{r_i}{s-\lambda_i} + r_0\right) \cdot U(s)
\]
Für die Matrix bedeutet dies:
\[
	\dot x=
	\underbrace{
		\begin{bmatrix}
			\lambda_1 &	 		  &  		& 	& & & \\
			1 		  & \lambda_1 &  		&  	& & &\\
			 		  & \ddots 	  & \ddots 	& 	& & &\\
				  	  & 	 	  & 1 		& \lambda_1 & & & \\
			 	      &  		  &  		&  	& \lambda_{m+1} & &\\	
			 	      &  		  &  		&  	& 				& \ddots &\\
			 	      &  		  &  		&  	& 				&  & \lambda_n\\
		\end{bmatrix}
	}_{\textbf{A}}
	\cdot x +
	\underbrace{
		\begin{bmatrix}
			1 \\
			0 \\
			\vdots \\
			0 \\
			1\\
			\vdots\\
			1	
		\end{bmatrix}
	}_{\textbf{b}}
	\cdot u	
\]

\[
	y=
	\underbrace{
			\begin{bmatrix}
				r_1 & \ldots & r_m & r_{m+1} & \ldots & r_n\\
			\end{bmatrix}
	}_{\textbf{$c^T$}}
	\cdot x  +
	\underbrace{
		\left[ r_0 \right] 
	}_{\textbf{d}}
	\cdot u
\]